% ********** Rozdział 4 **********
\chapter{Podsumowanie}

W niniejszym rozdziale przedstawiono zrealizowane prace związane z projektem oraz kierunki dalszego rozwoju systemu.

\section{Zrealizowane prace}
W ramach projektu udało się zrealizować następujące elementy:
\begin{itemize}
    \item Stworzenie wielowarstwowej architektury aplikacji, obejmującej warstwę dostępu do danych oraz interfejs użytkownika.
    \item Implementacja systemu zarządzania pojazdami, w tym obsługa dodawania, usuwania, edycji rejestracji, wyszukiwania oraz wyświetlania pojazdów.
    \item Integracja danych z dwóch źródeł – bazy plikowej i bazy SQL – umożliwiająca eksport danych oraz synchronizację między nimi.
    \item Opracowanie konsolowego interfejsu użytkownika, który zapewnia intuicyjną obsługę systemu poprzez menu oraz wizualizację operacji (poprzez dołączone zrzuty ekranu).
    \item Zabezpieczenie operacji na danych poprzez implementację odpowiednich walidacji oraz obsługi wyjątków.
\end{itemize}

\section{Planowane dalsze prace rozwojowe}
W kolejnych etapach rozwoju projektu planowane są następujące działania:
\begin{itemize}
    \item Rozszerzenie funkcjonalności interfejsu użytkownika poprzez wdrożenie graficznego interfejsu (GUI), co umożliwi bardziej przyjazną interakcję z systemem.
    \item Ulepszenie mechanizmu walidacji danych i obsługi błędów, aby zwiększyć niezawodność aplikacji w środowisku produkcyjnym.
    \item Integracja systemu z nowoczesnymi rozwiązaniami bazodanowymi oraz API, co pozwoli na lepszą skalowalność i integrację z innymi systemami.
    \item Przeprowadzenie testów wydajnościowych oraz optymalizacja operacji na dużych zbiorach danych, aby zapewnić szybki dostęp do informacji.
    \item Rozwój modułu raportowania i analizy statystyk, co umożliwi generowanie szczegółowych raportów oraz lepsze podejmowanie decyzji na podstawie zgromadzonych danych.
\end{itemize}


% ********** Koniec rozdziału **********

